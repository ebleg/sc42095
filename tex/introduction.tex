Adequate attitude control is crucial for space stations to function properly. For example, the solar arrays must be kept at the correct angle with respect to the Sun, but also instruments and antennae are highly dependent on the orientation of the spacecraft to perform their intended tasks.

In this report, an orientation controller will be designed for a space station. The space station is modelled as a SISO continuous transfer function. Since most controllers these days are implemented digitally, the aim of this report is to develop an adequate discrete-time controller for the space station.

The structure of the report is the following. \Cref{sec:continuous} treat the design of continuous-time PID controllers, respectively for reference tracking (servo problem) and load disturbance rejection (regulation problem). Subsequently, the controller designs and the plant are discretised in \cref{sec:discretise}. 

In \cref{sec:ss}, the controller design is approached from a state-space perspective; more specifically using pole-placement and the LQR method.
For all these controllers, the main priorities are zero steady-state error and minimal settling time within a set overshoot limit. The resulting controller designs are therefore very aggressive (and not entirely realistic). This issue is be addressed in \cref{sec:retune}; the control effort associated with the controller designs is first determined. It is given that the actuators will saturate at an input of $\pm20$, the controller that violate this bound are subsequently retuned to remain within these bounds.

\Cref{sec:q12} formalises the methods used to eliminate steady-state errors; as several have been used for the various controller designs. 

Finally, in \cref{sec:delay}, the controllers are again readjusted for the presence of a delay of one time step on the control input.

\subsection*{TODO}
\begin{itemize}
    \item Iets over observability bij ouput feedback
    \item Checken of alle pole locations duidelijk zijn gedocumenteerd
    \item Checken of alle sampling periods duidelijk zijn gedocumenteerd
    \item Feedback laws goed documenteren
\end{itemize}